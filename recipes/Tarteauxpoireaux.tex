\newpage
\section{Tarte aux poireaux}
    \label{sec:Tarte aux poireaux}
    \subsection{Recette}
    \vspace{1cm}

%    \begin{chapquote}{[quoteAuthor], \textit{[quoteAuthorText]}}
%        ``[Quote]''
%    \end{chapquote}

    \begin{center}
        \begin{tabular}{c|c}
            Temps de préparation & <= 45 \\
            Difficulté & 2 \\
            Nombre de personnes & 6 
        \end{tabular}
    \end{center}{}

    \vspace{1cm}
    \hline
    \vspace{1cm}

    \begin{minipage}{.7\textwidth}
        \begin{enumerate}
            \item Abaisser la p�te dans un moule � tarte
	    \item Piquer le fond
	    \item Faire pr�cuire le fond de tarte 7 minutes au four � 200�C
	    \item Eplucher et laver les poireaux en enlevant les deux tiers de la partie verte
	    \item Emincer les poireaux finement
	    \item Les faire revenir � feu doux avec une noix de beurre
	    \item Saler et poivrer
	    \item R�partir les poireaux dans le fond de tarte pr�cuit
	    \item Les saupoudrer de gruy�re
	    \item Mixer le lait, la cr�me fra�che, les oeufs, la farine, une pinc�e de sel et de poivre
	    \item Verser le m�lange sur les poireaux
	    \item Cuire au four pendant 20 minutes � 200�C.

        \end{enumerate}
    \end{minipage}
    \begin{minipage}{.3\textwidth}
        \begin{flushleft}
        \begin{itemize}
            \item 1 p�te bris�e
	    \item 800g de poireaux
	    \item 3 oeufs
	    \item 150 mL de cr�me fra�che
	    \item 70g de gruy�re r�p�
	    \item 200 mL de lait
	    \item 1 cuill�re � caf� de ma�zena

        \end{itemize}
        \end{flushleft}
    \end{minipage}
    
    \vspace{1cm}
    \hline
    \vspace{1cm}
    
    \subsection{Trucs \& Astuces}
        \subsection{Trucs \& Astuces}
	Servir avec une bi�re bien fra�che!
