\newpage
\section{Hachis Parmentier}
    \label{sec:Hachis Parmentier}
    \subsection{Recette}
    \vspace{1cm}

%    \begin{chapquote}{[quoteAuthor], \textit{[quoteAuthorText]}}
%        ``[Quote]''
%    \end{chapquote}

    \begin{center}
        \begin{tabular}{c|c}
            Temps de préparation & <= 45 \\
            Difficulté & 3 \\
            Nombre de personnes & 6 
        \end{tabular}
    \end{center}{}

    \vspace{1cm}
    \hline
    \vspace{1cm}

    \begin{minipage}{.7\textwidth}
        \begin{enumerate}
            \item Faire revenir les oignons avec le beurre jusqu'� coloration sur feu moyen
	    \item Ajouter la farine, puis d�layer avec le bouillon, le vin et la pur�e de tomates
	    \item Porter � �bullition pendant une dizaine de minutes
	    \item Ajouter la viande hach�e, bien m�langer
	    \item �taler le m�lange dans un plat et y superposer la pur�e de pommes de terre
	    \item Ajouter l'emmental r�p� et faire gratiner au four.

        \end{enumerate}
    \end{minipage}
    \begin{minipage}{.3\textwidth}
        \begin{flushleft}
        \begin{itemize}
            \item Pur�e de pomme de terre
	    \item 500g de boeuf cuit
	    \item 120g d'oignons hach�s
	    \item 40g de beurre
	    \item 20-25g de farine
	    \item 200mL de vin blanc
	    \item 200mL de bouillon d�graiss�
	    \item 1 cuill�re � soupe de pur�e de tomates
	    \item 1 cuill�re � caf� de persil hach�
	    \item emmental r�p�

        \end{itemize}
        \end{flushleft}
    \end{minipage}
    
    \vspace{1cm}
    \hline
    \vspace{1cm}
    
    \subsection{Trucs \& Astuces}
        \subsection{Trucs \& Astuces}
	Utiliser un reste de viande et de bouillon provenant d'un pot au feu!
