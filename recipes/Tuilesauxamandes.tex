\newpage
\section{Tuiles aux amandes}
    \label{sec:Tuiles aux amandes}
    \subsection{Recette}
    \vspace{1cm}

%    \begin{chapquote}{[quoteAuthor], \textit{[quoteAuthorText]}}
%        ``[Quote]''
%    \end{chapquote}

    \begin{center}
        \begin{tabular}{c|c}
            Temps de préparation & <= 30min \\
            Difficulté & 1 \\
            Nombre de personnes & 6 
        \end{tabular}
    \end{center}{}

    \vspace{1cm}
    \hline
    \vspace{1cm}

    \begin{minipage}{.7\textwidth}
        \begin{enumerate}
            \item M�langer les amandes effil�es, le sucre, la farine
	    \item Y ajouter le beurre fondu puis la cr�me fra�che
	    \item Faire plusieurs petites boules sur une plaque
	    \item Cuire quelques minutes � 180�C.

        \end{enumerate}
    \end{minipage}
    \begin{minipage}{.3\textwidth}
        \begin{flushleft}
        \begin{itemize}
            \item 125g d'amandes effil�es
	    \item 100g de beurre
	    \item 100g de sucre
	    \item 2 cuill�res � soupe de farine
	    \item 1 cuill�re � soupe de cr�me fra�che

        \end{itemize}
        \end{flushleft}
    \end{minipage}
    
    \vspace{1cm}
    \hline
    \vspace{1cm}
    
    \subsection{Trucs \& Astuces}
        \subsection{Trucs \& Astuces}
	Bien �carte les boules sur la plaque de cuisson sinon �a colle!
