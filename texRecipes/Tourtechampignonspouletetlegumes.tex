\newpage
\section{Tourte champignons poulet et légumes}
    \label{sec:Tourte champignons poulet et légumes}
    \subsection{Recette}
    \vspace{1cm}

%    \begin{chapquote}{[quoteAuthor], \textit{[quoteAuthorText]}}
%        ``[Quote]''
%    \end{chapquote}

    \begin{center}
        \begin{tabular}{c|c}
            Temps de préparation & >= 1h \\
            Difficulté & 3.0 \\
            Nombre de personnes & 6.0 
        \end{tabular}
    \end{center}{}

    \vspace{1cm}
    \hline
    \vspace{1cm}

    \begin{minipage}{.7\textwidth}
        \begin{enumerate}
            \item Préchauffez le four à 200]C et beurrez le moule (possible en ramequins)
	    \item Nettoyez les oignons et coupez les
	    \item Pelez les carottes et taillez les en dés
	    \item Nettoyez et coupez les champignons en morceaux de taille moyenne
	    \item Poêlez le poulet, coupé en dés, avec de l'huile d'olive, sel et poivre, puis réservez
	    \item Faites fondre les oignons dans de l'huile d'olive à la poêle puis ajoutez les carottes et laissez les caraméliser
	    \item Ajoutez les champignons et l'ail émincé, avec du romarin haché, assaisonnez et laisser cuire
	    \item Ajoutez finalement la farine et mélangez bien
	    \item Une fois cuit, versez progressivement le bouillon, puis ajoutez la crème fraîche et le poulet
	    \item Tapissez le moule d'une première pâte, puis ajoutez y la préparation et recouvrez de la seconde pâte
	    \item Froncez les deux pâtes ensembles
	    \item Badigeonnez de jaune d'oeuf et parsemez de gruyère avant d'enfourner 20minutes.

        \end{enumerate}
    \end{minipage}
    \begin{minipage}{.3\textwidth}
        \begin{flushleft}
        \begin{itemize}
            \item 2 pâtes feuilletées
	    \item 250 de champignons rosés de Paris
	    \item 250g de blanc de poulet
	    \item 6 petits oignons blancs frais
	    \item 2 carottes
	    \item 3 cuillères à soupe d'huile d'olive
	    \item 3 gousses d'ail
	    \item Romarin frais
	    \item 3 cuillères à soupe de farine
	    \item 500mL de bouillon de volaille
	    \item 4 cuillères à soupe de crème fraîche épaisse
	    \item 1 jaune d'oeuf
	    \item 40g de gruyère râpé
	    \item beurre

        \end{itemize}
        \end{flushleft}
    \end{minipage}
    
    \vspace{1cm}
    \hline
    \vspace{1cm}
    
    \subsection{Trucs \& Astuces}
        \subsection{Trucs \& Astuces}
	N'hésitez pas à laisser l'eau des champignons s'évaporer et/ou à rajouter un peu de farine afin d'obtenir une préparation liée de façon dense.
