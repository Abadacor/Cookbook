\newpage
\section{Financiers}
    \label{sec:Financiers}
    \subsection{Recette}
    \vspace{1cm}

%    \begin{chapquote}{[quoteAuthor], \textit{[quoteAuthorText]}}
%        ``[Quote]''
%    \end{chapquote}

    \begin{center}
        \begin{tabular}{c|c}
            Temps de préparation & <= 30min \\
            Difficulté & 1 \\
            Nombre de personnes & 6 
        \end{tabular}
    \end{center}{}

    \vspace{1cm}
    \hline
    \vspace{1cm}

    \begin{minipage}{.7\textwidth}
        \begin{enumerate}
            \item Préchauffer le four à 190°C
	    \item Faire fondre le beurre
	    \item Mélanger le sucre, la poudre d'amande et les oeufs
	    \item Verser le beurre fondu et bien mélanger
	    \item Ajouter la farine
	    \item Beurrer les moules et les remplir de pâte
	    \item  Enfourner pour 12 minutes.

        \end{enumerate}
    \end{minipage}
    \begin{minipage}{.3\textwidth}
        \begin{flushleft}
        \begin{itemize}
            \item 125g de sucre
	    \item 125g de poudre d'amande
	    \item 70g de beurre salé
	    \item 2 oeufs
	    \item 20g de farine

        \end{itemize}
        \end{flushleft}
    \end{minipage}
    
    \vspace{1cm}
    \hline
    \vspace{1cm}
    
    \subsection{Trucs \& Astuces}
        \subsection{Trucs \& Astuces}
	Ajouter du thé vert, des épices ou des framboises pour varier les plaisirs!
