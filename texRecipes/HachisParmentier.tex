\newpage
\section{Hachis Parmentier}
    \label{sec:Hachis Parmentier}
    \subsection{Recette}
    \vspace{1cm}

%    \begin{chapquote}{[quoteAuthor], \textit{[quoteAuthorText]}}
%        ``[Quote]''
%    \end{chapquote}

    \begin{center}
        \begin{tabular}{c|c}
            Temps de préparation & <= 45 \\
            Difficulté & 3 \\
            Nombre de personnes & 6 
        \end{tabular}
    \end{center}{}

    \vspace{1cm}
    \hline
    \vspace{1cm}

    \begin{minipage}{.7\textwidth}
        \begin{enumerate}
            \item Faire revenir les oignons avec le beurre jusqu'à coloration sur feu moyen
	    \item Ajouter la farine, puis délayer avec le bouillon, le vin et la purée de tomates
	    \item Porter à ébullition pendant une dizaine de minutes
	    \item Ajouter la viande hachée, bien mélanger
	    \item Étaler le mélange dans un plat et y superposer la purée de pommes de terre
	    \item Ajouter l'emmental râpé et faire gratiner au four.

        \end{enumerate}
    \end{minipage}
    \begin{minipage}{.3\textwidth}
        \begin{flushleft}
        \begin{itemize}
            \item Purée de pomme de terre
	    \item 500g de boeuf cuit
	    \item 120g d'oignons hachés
	    \item 40g de beurre
	    \item 20-25g de farine
	    \item 200mL de vin blanc
	    \item 200mL de bouillon dégraissé
	    \item 1 cuillère à soupe de purée de tomates
	    \item 1 cuillère à café de persil haché
	    \item emmental râpé

        \end{itemize}
        \end{flushleft}
    \end{minipage}
    
    \vspace{1cm}
    \hline
    \vspace{1cm}
    
    \subsection{Trucs \& Astuces}
        \subsection{Trucs \& Astuces}
	Utiliser un reste de viande et de bouillon provenant d'un pot au feu!
