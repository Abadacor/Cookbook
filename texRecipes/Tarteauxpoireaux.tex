\newpage
\section{Tarte aux poireaux}
    \label{sec:Tarte aux poireaux}
    \subsection{Recette}
    \vspace{1cm}

%    \begin{chapquote}{[quoteAuthor], \textit{[quoteAuthorText]}}
%        ``[Quote]''
%    \end{chapquote}

    \begin{center}
        \begin{tabular}{c|c}
            Temps de préparation & <= 45 \\
            Difficulté & 2 \\
            Nombre de personnes & 6 
        \end{tabular}
    \end{center}{}

    \vspace{1cm}
    \hline
    \vspace{1cm}

    \begin{minipage}{.7\textwidth}
        \begin{enumerate}
            \item Abaisser la pâte dans un moule à tarte
	    \item Piquer le fond
	    \item Faire précuire le fond de tarte 7 minutes au four à 200°C
	    \item Eplucher et laver les poireaux en enlevant les deux tiers de la partie verte
	    \item Emincer les poireaux finement
	    \item Les faire revenir à feu doux avec une noix de beurre
	    \item Saler et poivrer
	    \item Répartir les poireaux dans le fond de tarte précuit
	    \item Les saupoudrer de gruyère
	    \item Mixer le lait, la crème fraîche, les oeufs, la farine, une pincée de sel et de poivre
	    \item Verser le mélange sur les poireaux
	    \item Cuire au four pendant 20 minutes à 200°C.

        \end{enumerate}
    \end{minipage}
    \begin{minipage}{.3\textwidth}
        \begin{flushleft}
        \begin{itemize}
            \item 1 pâte brisée
	    \item 800g de poireaux
	    \item 3 oeufs
	    \item 150 mL de crème fraîche
	    \item 70g de gruyère râpé
	    \item 200 mL de lait
	    \item 1 cuillère à café de maïzena

        \end{itemize}
        \end{flushleft}
    \end{minipage}
    
    \vspace{1cm}
    \hline
    \vspace{1cm}
    
    \subsection{Trucs \& Astuces}
        \subsection{Trucs \& Astuces}
	Servir avec une bière bien fraîche!
