\newpage
\section{Tourte champignons poulet et l�gumes}
    \label{sec:Tourte champignons poulet et l�gumes}
    \subsection{Recette}
    \vspace{1cm}

%    \begin{chapquote}{[quoteAuthor], \textit{[quoteAuthorText]}}
%        ``[Quote]''
%    \end{chapquote}

    \begin{center}
        \begin{tabular}{c|c}
            Temps de préparation & >= 1h \\
            Difficulté & 3 \\
            Nombre de personnes & 6 
        \end{tabular}
    \end{center}{}

    \vspace{1cm}
    \hline
    \vspace{1cm}

    \begin{minipage}{.7\textwidth}
        \begin{enumerate}
            \item Pr�chauffez le four � 200]C et beurrez le moule (possible en ramequins)
	    \item Nettoyez les oignons et coupez les
	    \item Pelez les carottes et taillez les en d�s
	    \item Nettoyez et coupez les champignons en morceaux de taille moyenne
	    \item Po�lez le poulet, coup� en d�s, avec de l'huile d'olive, sel et poivre, puis r�servez
	    \item Faites fondre les oignons dans de l'huile d'olive � la po�le puis ajoutez les carottes et laissez les caram�liser
	    \item Ajoutez les champignons et l'ail �minc�, avec du romarin hach�, assaisonnez et laisser cuire
	    \item Ajoutez finalement la farine et m�langez bien
	    \item Une fois cuit, versez progressivement le bouillon, puis ajoutez la cr�me fra�che et le poulet
	    \item Tapissez le moule d'une premi�re p�te, puis ajoutez y la pr�paration et recouvrez de la seconde p�te
	    \item Froncez les deux p�tes ensembles
	    \item Badigeonnez de jaune d'oeuf et parsemez de gruy�re avant d'enfourner 20minutes.

        \end{enumerate}
    \end{minipage}
    \begin{minipage}{.3\textwidth}
        \begin{flushleft}
        \begin{itemize}
            \item 2 p�tes feuillet�es
	    \item 250 de champignons ros�s de Paris
	    \item 250g de blanc de poulet
	    \item 6 petits oignons blancs frais
	    \item 2 carottes
	    \item 3 cuill�res � soupe d'huile d'olive
	    \item 3 gousses d'ail
	    \item Romarin frais
	    \item 3 cuill�res � soupe de farine
	    \item 500mL de bouillon de volaille
	    \item 4 cuill�res � soupe de cr�me fra�che �paisse
	    \item 1 jaune d'oeuf
	    \item 40g de gruy�re r�p�
	    \item beurre

        \end{itemize}
        \end{flushleft}
    \end{minipage}
    
    \vspace{1cm}
    \hline
    \vspace{1cm}
    
    \subsection{Trucs \& Astuces}
        \subsection{Trucs \& Astuces}
	N'h�sitez pas � laisser l'eau des champignons s'�vaporer et/ou � rajouter un peu de farine afin d'obtenir une pr�paration li�e de fa�on dense.
