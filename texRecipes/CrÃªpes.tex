\newpage
\section{Cr�pes}
    \label{sec:Cr�pes}
    \subsection{Recette}
    \vspace{1cm}

%    \begin{chapquote}{[quoteAuthor], \textit{[quoteAuthorText]}}
%        ``[Quote]''
%    \end{chapquote}

    \begin{center}
        \begin{tabular}{c|c}
            Temps de préparation & <= 15min \\
            Difficulté & 1 \\
            Nombre de personnes & 6 
        \end{tabular}
    \end{center}{}

    \vspace{1cm}
    \hline
    \vspace{1cm}

    \begin{minipage}{.7\textwidth}
        \begin{enumerate}
            \item Tout m�langer jusqu'� obtention d'une p�te lisse
	    \item 12 cr�pes obtenables avec ces quantit�s.

        \end{enumerate}
    \end{minipage}
    \begin{minipage}{.3\textwidth}
        \begin{flushleft}
        \begin{itemize}
            \item 125g de farine
	    \item 3 oeufs
	    \item 30g de beurre (pour la po�le)
	    \item 250mL de lait
	    \item Fleur d'oranger
	    \item Un bouchon de rhum
	    \item sel

        \end{itemize}
        \end{flushleft}
    \end{minipage}
    
    \vspace{1cm}
    \hline
    \vspace{1cm}
    
    \subsection{Trucs \& Astuces}
        \subsection{Trucs \& Astuces}
	Si mixeur ou plongeur disponible, � utiliser. Pour de meilleures cr�pes laisser la p�te reposer 1 � 2 heures.
