\newpage
\section{Cake salé (chèvre courgettes)}
    \label{sec:Cake salé (chèvre courgettes)}
    \subsection{Recette}
    \vspace{1cm}

%    \begin{chapquote}{Cécile, \textit{Soucieuse de voir son talent culinaire reconnu}}
%        ``egin{chapquote}{Cécile, 	extit{Soucieuse de voir son talent culinaire reconnu}}
\`\`Ah! Celui là je sais faire!
''
\end{chapquote}
''
%    \end{chapquote}

    \begin{center}
        \begin{tabular}{c|c}
            Temps de préparation & >= 1h \\
            Difficulté & 2 \\
            Nombre de personnes & 6 
        \end{tabular}
    \end{center}{}

    \vspace{1cm}
    \hline
    \vspace{1cm}

    \begin{minipage}{.7\textwidth}
        \begin{enumerate}
            \item Pelez la courgette en rayures et coupez là en dés
	    \item Rissolez là à la poêle avec un fond d'huile, sel, poivre et paprika
	    \item Coupez le chèvre en dés
	    \item Mélangez fermement le reste des ingrédients jusqu'à obtention d'une pâte homogène
	    \item Ajoutez-y les courgettes et le chèvre
	    \item Versez la préparation dans un moule à cake non beurré
	    \item Enfournez pendant 45 minutes environ dans un four à 180°C.

        \end{enumerate}
    \end{minipage}
    \begin{minipage}{.3\textwidth}
        \begin{flushleft}
        \begin{itemize}
            \item 3 oeufs
	    \item 150g de farine
	    \item 1 sahcet de levure
	    \item 80mL d'huile
	    \item 125mL de lait
	    \item 100g de gruyère râpé
	    \item 200g de chèvre
	    \item 1 courgette
	    \item herbes arômatiques à convenance
	    \item paprika

        \end{itemize}
        \end{flushleft}
    \end{minipage}
    
    \vspace{1cm}
    \hline
    \vspace{1cm}
    
    \subsection{Trucs \& Astuces}
        \subsection{Trucs \& Astuces}
	Recette à varier le plus possible. Utiliser un légume cuit et un ou deux condiments. 
