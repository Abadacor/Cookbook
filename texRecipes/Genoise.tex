\newpage
\section{Génoise}
    \label{sec:Génoise}
    \subsection{Recette}
    \vspace{1cm}

%    \begin{chapquote}{[quoteAuthor], \textit{[quoteAuthorText]}}
%        ``[Quote]''
%    \end{chapquote}

    \begin{center}
        \begin{tabular}{c|c}
            Temps de préparation & <= 15min \\
            Difficulté & 1.0 \\
            Nombre de personnes & 6.0 
        \end{tabular}
    \end{center}{}

    \vspace{1cm}
    \hline
    \vspace{1cm}

    \begin{minipage}{.7\textwidth}
        \begin{enumerate}
            \item Précauffer le four à 175°C
	    \item Mélanger jaunes d'oeufs, sucre et sucre vanillé
	    \item Ajouter farine, maïzena et levure préalablement mélangés
	    \item Battre les blancs en neige
	    \item Ajouter à la préparation délicatement
	    \item Beurrer le moule
	    \item Verser la pâte et faire cuire 35 minutes.

        \end{enumerate}
    \end{minipage}
    \begin{minipage}{.3\textwidth}
        \begin{flushleft}
        \begin{itemize}
            \item 30g de farine
	    \item 1 sachet de sucre vanillé
	    \item 1 sachet de levure
	    \item 1 pincée de sel
	    \item 70g de maïzena
	    \item 5 oeufs
	    \item 125g de sucre

        \end{itemize}
        \end{flushleft}
    \end{minipage}
    
    \vspace{1cm}
    \hline
    \vspace{1cm}
    
    \subsection{Trucs \& Astuces}
        \subsection{Trucs \& Astuces}
	Ne pas ouvrir le four pendant la cuisson pour éviter de faire dégonfler la génoise
