\newpage
\section{Gâteau au chocolat}
    \label{sec:Gâteau au chocolat}
    \subsection{Recette}
    \vspace{1cm}

%    \begin{chapquote}{Adèle, \textit{Un grand sourire sur les lèvres}}
%        ``egin{chapquote}{Adèle, 	extit{Un grand sourire sur les lèvres}}
\`\`J'ai monté de la crème anglaise en plus!
''
\end{chapquote}
''
%    \end{chapquote}

    \begin{center}
        \begin{tabular}{c|c}
            Temps de préparation & <= 45 \\
            Difficulté & 2.0 \\
            Nombre de personnes & 6.0 
        \end{tabular}
    \end{center}{}

    \vspace{1cm}
    \hline
    \vspace{1cm}

    \begin{minipage}{.7\textwidth}
        \begin{enumerate}
            \item Faire fondre le chocolat et le beurre sur feu doux
	    \item Séparer les oeufs et battre les jaunes avant d'y incorporer la préparation précédente
	    \item Y ajouter la purée de marrons
	    \item Faire monter les blancs en neige avant de les incorporer au reste de la réparation
	    \item Enfourner pendant 30 minutes à 180°C.

        \end{enumerate}
    \end{minipage}
    \begin{minipage}{.3\textwidth}
        \begin{flushleft}
        \begin{itemize}
            \item 5 oeufs
	    \item 200g de chocolat noir
	    \item 150g de beurre
	    \item 450g de crème de marrons

        \end{itemize}
        \end{flushleft}
    \end{minipage}
    
    \vspace{1cm}
    \hline
    \vspace{1cm}
    
    \subsection{Trucs \& Astuces}
        \subsection{Trucs \& Astuces}
	Attention à ne pas casser les blancs. Personne ne sait ce que ça veut dire, mais c'est un bon conseil.
