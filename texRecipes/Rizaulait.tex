\newpage
\section{Riz au lait}
    \label{sec:Riz au lait}
    \subsection{Recette}
    \vspace{1cm}

%    \begin{chapquote}{Romain, \textit{Regardant son père engloutir sa part}}
%        ``egin{chapquote}{Romain, 	extit{Regardant son père engloutir sa part}}
\`\`Pourquoi il se ressert, il est pas sensé ne pas aimer ça?
''
\end{chapquote}
''
%    \end{chapquote}

    \begin{center}
        \begin{tabular}{c|c}
            Temps de préparation & <= 45 \\
            Difficulté & 3 \\
            Nombre de personnes & 4 
        \end{tabular}
    \end{center}{}

    \vspace{1cm}
    \hline
    \vspace{1cm}

    \begin{minipage}{.7\textwidth}
        \begin{enumerate}
            \item Faire cuire le riz 5min dans de l'eau, puis égoutter
	    \item Faire bouillir le lait avec le surcre vanillé, le sel et le zeste de citron
	    \item Rajouter le riz une fois le mélange parvenu à ébullition
	    \item Cuire 25 à 35minutes 
	    \item Rajouter le jaune d'oeuf tout en remuant afin de ne pas le cuire, puis le sucre et le beurre.

        \end{enumerate}
    \end{minipage}
    \begin{minipage}{.3\textwidth}
        \begin{flushleft}
        \begin{itemize}
            \item 200g de riz à cuire
	    \item 750mL de lait
	    \item 1 sachet de sucre vanillé
	    \item 1 pincée de sel
	    \item du zeste de citron
	    \item 1 jaune d'oeuf
	    \item 125g de sucre
	    \item 1 noix de beurre

        \end{itemize}
        \end{flushleft}
    \end{minipage}
    
    \vspace{1cm}
    \hline
    \vspace{1cm}
    
    \subsection{Trucs \& Astuces}
        \subsection{Trucs \& Astuces}
	Utiliser si possible du riz rond. Sortir le mélange du feu une minute avant d'incorporer le jaune d'oeuf et bien remuer pendant.
