\newpage
\section{G�teau au chocolat}
    \label{sec:G�teau au chocolat}
    \subsection{Recette}
    \vspace{1cm}

%    \begin{chapquote}{Ad�le, \textit{Un grand sourire sur les l�vres}}
%        ``egin{chapquote}{Ad�le, 	extit{Un grand sourire sur les l�vres}}
\`\`J'ai mont� de la cr�me anglaise en plus!
''
\end{chapquote}
''
%    \end{chapquote}

    \begin{center}
        \begin{tabular}{c|c}
            Temps de préparation & <= 45 \\
            Difficulté & 2 \\
            Nombre de personnes & 6 
        \end{tabular}
    \end{center}{}

    \vspace{1cm}
    \hline
    \vspace{1cm}

    \begin{minipage}{.7\textwidth}
        \begin{enumerate}
            \item Faire fondre le chocolat et le beurre sur feu doux
	    \item S�parer les oeufs et battre les jaunes avant d'y incorporer la pr�paration pr�c�dente
	    \item Y ajouter la pur�e de marrons
	    \item Faire monter les blancs en neige avant de les incorporer au reste de la r�paration
	    \item Enfourner pendant 30 minutes � 180�C.

        \end{enumerate}
    \end{minipage}
    \begin{minipage}{.3\textwidth}
        \begin{flushleft}
        \begin{itemize}
            \item 5 oeufs
	    \item 200g de chocolat noir
	    \item 150g de beurre
	    \item 450g de cr�me de marrons

        \end{itemize}
        \end{flushleft}
    \end{minipage}
    
    \vspace{1cm}
    \hline
    \vspace{1cm}
    
    \subsection{Trucs \& Astuces}
        \subsection{Trucs \& Astuces}
	Attention � ne pas casser les blancs. Personne ne sait ce que �a veut dire, mais c'est un bon conseil.
