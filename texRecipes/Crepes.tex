\newpage
\section{Crêpes}
    \label{sec:Crêpes}
    \subsection{Recette}
    \vspace{1cm}

%    \begin{chapquote}{[quoteAuthor], \textit{[quoteAuthorText]}}
%        ``[Quote]''
%    \end{chapquote}

    \begin{center}
        \begin{tabular}{c|c}
            Temps de préparation & <= 15min \\
            Difficulté & 1.0 \\
            Nombre de personnes & 6.0 
        \end{tabular}
    \end{center}{}

    \vspace{1cm}
    \hline
    \vspace{1cm}

    \begin{minipage}{.7\textwidth}
        \begin{enumerate}
            \item Tout mélanger jusqu'à obtention d'une pâte lisse
	    \item 12 crêpes obtenables avec ces quantités.

        \end{enumerate}
    \end{minipage}
    \begin{minipage}{.3\textwidth}
        \begin{flushleft}
        \begin{itemize}
            \item 125g de farine
	    \item 3 oeufs
	    \item 30g de beurre (pour la poêle)
	    \item 250mL de lait
	    \item Fleur d'oranger
	    \item Un bouchon de rhum
	    \item sel

        \end{itemize}
        \end{flushleft}
    \end{minipage}
    
    \vspace{1cm}
    \hline
    \vspace{1cm}
    
    \subsection{Trucs \& Astuces}
        \subsection{Trucs \& Astuces}
	Si mixeur ou plongeur disponible, à utiliser. Pour de meilleures crêpes laisser la pâte reposer 1 à 2 heures.
