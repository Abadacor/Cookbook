\newpage
\section{Gougères}
    \label{sec:Gougères}
    \subsection{Recette}
    \vspace{1cm}

%    \begin{chapquote}{Romain, \textit{Regrettant sa décision de faire des gougères}}
%        ``egin{chapquote}{Romain, 	extit{Regrettant sa décision de faire des gougères}}
\`\`J'ai mal au bras..
''
\end{chapquote}
''
%    \end{chapquote}

    \begin{center}
        \begin{tabular}{c|c}
            Temps de préparation & <= 30min \\
            Difficulté & 4 \\
            Nombre de personnes & 6 
        \end{tabular}
    \end{center}{}

    \vspace{1cm}
    \hline
    \vspace{1cm}

    \begin{minipage}{.7\textwidth}
        \begin{enumerate}
            \item Dans une casserole, faire bouillir l'eau, le beurre et le sel
	    \item Au premier bouillon retirer du feu, et y ajouter la farine
	    \item Remuer sur le feu jusqu'à ce que la pâte se décolle
	    \item Incorporer les oeufs un à un
	    \item Ajouter 200g de gruyère
	    \item Séparer la pâte obtenue en choux et faire cuire au four à 210°C pendant 20 minutes.

        \end{enumerate}
    \end{minipage}
    \begin{minipage}{.3\textwidth}
        \begin{flushleft}
        \begin{itemize}
            \item 300mL d'eau
	    \item 125g de beurre
	    \item 1 pincée de sel
	    \item 250g de farine
	    \item 5-6 oeufs
	    \item 200g de gruyère

        \end{itemize}
        \end{flushleft}
    \end{minipage}
    
    \vspace{1cm}
    \hline
    \vspace{1cm}
    
    \subsection{Trucs \& Astuces}
        \subsection{Trucs \& Astuces}
	Au moment de l'incorporation d'un oeuf, retirer du feu, puis remettre une fois homogène.
